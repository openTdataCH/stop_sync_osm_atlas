% !TeX spellcheck = fr_FR
\chapter{Chapitre 3 : Correspondance avec les données ATLAS-OSM}

Le processus de correspondance entre les données ATLAS et OSM a été conçu pour identifier de manière précise et systématique les arrêts correspondants dans ces deux ensembles de données. Cette approche méthodologique repose sur un principe d'appariement séquentiel par ordre de fiabilité.

\section{Approche méthodologique générale}

Le processus de correspondance adopte une stratégie «\ hit-first\ » (première correspondance trouvée), où chaque entrée ATLAS est appariée selon la première méthode qui réussit, en suivant un ordre décroissant de fiabilité. Cette approche séquentielle garantit que les correspondances les plus fiables (exactes) sont privilégiées par rapport aux correspondances moins certaines (par distance).

Nous avons envisagé une approche alternative consistant à exécuter toutes les méthodes sur toutes les entrées, afin d'analyser pour chaque correspondance le nombre de méthodes qui fonctionnent et d'attribuer une probabilité de correspondance. Cependant, après réflexion approfondie, nous avons conclu que cette approche n'apporterait pas de valeur ajoutée significative. En effet, l'ordre séquentiel reflète déjà la hiérarchie de fiabilité : si une correspondance exacte est trouvée, il est inutile de vérifier si d'autres méthodes moins fiables fonctionnent également.

Le processus complet comprend les étapes suivantes :
\begin{enumerate}
    \item \textbf{Appariement exact} : Appariement basé sur les identifiants UIC et la référence locale.
    \item \textbf{Correspondance par nom} : Utilisation des noms officiels des arrêts.
    \item \textbf{Correspondance par distance} : Analyse géographique avec critères de proximité.
    \item \textbf{Correspondance par routes} : Méthode complexe basée sur l'analyse des itinéraires (détaillée au chapitre 4).
    \item \textbf{Consolidation post-traitement} : Deuxième passage basé sur les identifiants UIC et les références locales avec les entrées restantes.
    \item \textbf{Propagation des duplicatas} : Si une entrée ATLAS est dupliquée, la correspondance est propagée à toutes les entrées dupliquées.
    \item \textbf{Correspondance manuelle} : Application des correspondances définies manuellement et stockées de manière persistante.
\end{enumerate}

Ce chapitre détaille les méthodes de correspondance exacte, par nom et par distance. La correspondance par routes, en raison de sa complexité particulière, sera traitée séparément au chapitre 4.


\section{Correspondance Exacte}

La première étape, appelée correspondance exacte, utilise l’identifiant UIC. Dans les données ATLAS, cet identifiant est représenté par la colonne \texttt{'number'}, tandis que dans OSM, il correspond à la balise \texttt{'uic\_ref'}. Une entrée ATLAS est appariée à un nœud OSM si son \texttt{'number'} est identique au \texttt{'uic\_ref'} du nœud OSM.

Des situations complexes peuvent survenir lorsque plusieurs entrées ATLAS partagent le même \texttt{'number'} (par exemple, plusieurs quais d’une même gare) ou lorsque plusieurs nœuds OSM possèdent le même \texttt{'uic\_ref'}. Pour résoudre ces cas, les règles suivantes sont appliquées :

\begin{enumerate}
    \item \textbf{Cas 1 : Plusieurs entrées ATLAS, un seul nœud OSM}  
    Si plusieurs entrées ATLAS partagent le même \texttt{'number'} et qu’un seul nœud OSM possède ce \texttt{'uic\_ref'}, toutes ces entrées ATLAS sont appariées à ce nœud OSM unique.

    \item \textbf{Cas 2 : Une entrée ATLAS, plusieurs nœuds OSM}  
    Si une seule entrée ATLAS a un \texttt{'number'} donné et que plusieurs nœuds OSM partagent ce \texttt{'uic\_ref'}, tous ces nœuds OSM sont appariées à cette entrée ATLAS unique.

    \item \textbf{Cas 3 : Plusieurs entrées ATLAS et plusieurs nœuds OSM}  
    Lorsque plusieurs entrées ATLAS et nœuds OSM partagent le même \texttt{'number'}/\texttt{'uic\_ref'}, une correspondance plus fine est réalisée en comparant la \texttt{'designation'} de l’entrée ATLAS (par exemple, le code du quai) avec la balise \texttt{'local\_ref'} du nœud OSM. Une correspondance est établie si ces valeurs sont identiques.
\end{enumerate}

Cette méthode a permis d'identifier 21\,124 correspondances exactes.
\begin{figure}[h] 
    \centering
    \includegraphics[width=0.7\textwidth]{../figures/correspondances/exact_Cornavin.png}
    \caption[Correspondances exactes à Genève-Cornavin]{Correspondances exactes à la gare de Genève-Cornavin. Les détails de l'arrêt de la voie 5 sont visibles sur l'image.}
    \label{fig:exact_cornavin}
\end{figure}

\section{Correspondance par Nom}

Pour les entrées ATLAS non appariées lors de l’étape précédente, une correspondance basée sur le nom est appliquée. Cette étape compare le nom officiel des arrêts, indiqué dans la colonne \texttt{'designationOfficial'} des données ATLAS, avec plusieurs balises de nom dans OSM : \texttt{'name'}, \texttt{'uic\_name'} et \texttt{'gtfs:name'}.

La règle principale établit une correspondance si le \texttt{'designationOfficial'} correspond exactement à l’une de ces balises OSM. Cependant, si plusieurs nœuds OSM présentent le même nom, un critère supplémentaire est utilisé : la balise \texttt{'local\_ref'} du nœud OSM est comparée à la \texttt{'designation'} de l’entrée ATLAS. Une correspondance est confirmée si ces valeurs sont identiques (en ignorant la casse).

Cette approche a permis d'ajouter 535 correspondances supplémentaires.
\begin{figure}[h]
    \centering
    \includegraphics[width=\textwidth]{../figures/correspondances/matched_name.png}
    \caption[Exemple de correspondance par nom]{Pour l'arrêt "Thônex, Sous-Moulin, D", on peut voir que, malgré une référence UIC différente, il est possible d'établir des correspondances grâce au nom.}
    \label{fig:name_matching_example}
\end{figure}


\section{Correspondance par Distance}

Pour les entrées ATLAS restantes, une correspondance basée sur la proximité géographique est mise en œuvre. Cette étape se divise en trois sous-étapes distinctes, chacune avec des critères spécifiques pour garantir des appariements fiables.

\subsection{Étape 1 : Correspondance de groupe basée sur la proximité}
Les entrées ATLAS et OSM sont regroupés selon les paires d’identifiants suivantes :
\begin{enumerate}
    \item \texttt{'number'} (ATLAS) et \texttt{'uic\_ref'} (OSM).
    \item \texttt{'designationOfficial'} (ATLAS) et \texttt{'uic\_name'} (OSM).
    \item \texttt{'designationOfficial'} (ATLAS) et \texttt{'name'} (OSM).
\end{enumerate}

Dans chaque groupe où le nombre d'entrées ATLAS est égal au nombre de nœuds OSM, une correspondance est tentée en associant chaque entrée ATLAS au nœud OSM le plus proche, à condition que cette association soit cohérente (c'est-à-dire que chaque nœud OSM soit également le plus proche de l'entrée ATLAS qui lui est attribuée). Cette méthode nous a permis de réaliser 15\,384 correspondances supplémentaires.

\begin{figure}[h] 
    \centering
    \includegraphics[width=0.7\textwidth]{../figures/correspondances/groupe_proximite.png}
    \caption[Correspondances – Münchenstein, Hofmatt]{Correspondances pour les arrêts de Münchenstein, Hofmatt. Malgré les divergences de \texttt{uic\_ref} et le manque de références locales, nous avons réussi à établir des correspondances grâce à la correspondance de groupe basée sur les distances.}
    \label{fig:group_proximity_munchenstein}
\end{figure}

\FloatBarrier

\begin{table}
\caption[Données ATLAS – Münchenstein, Hofmatt]{Données ATLAS pour les arrêts de Münchenstein, Hofmatt}
\label{tab:atlas_data}
\centering
\begin{tabular}{l l l l l}
\toprule
\texttt{sloid} & \texttt{number} & \texttt{designation} & \texttt{designationOfficial} \\
\midrule
ch:1:sloid:95:1:6 & 8500095 &  & Münchenstein, Hofmatt \\
ch:1:sloid:95:1:5 & 8500095 &  & Münchenstein, Hofmatt \\
ch:1:sloid:95:1:3 & 8500095 &  & Münchenstein, Hofmatt \\
ch:1:sloid:95:1:2 & 8500095 &  & Münchenstein, Hofmatt \\
ch:1:sloid:95:1:1 & 8500095 &  & Münchenstein, Hofmatt \\
\bottomrule
\end{tabular}
\end{table}

\begin{table}[h]
\caption[Données OSM – Münchenstein, Hofmatt]{Données OSM pour les arrêts de Münchenstein, Hofmatt}
\label{tab:osm_data}
\centering
\begin{tabular}{l l l l}
\toprule
\texttt{node\_id} & \texttt{uic\_ref} & \texttt{uic\_name} & \texttt{transport\_type} \\
\midrule
6457499611 & 8578185 & Münchenstein, Hofmatt & bus \\
299126238 & 8500095 & Münchenstein, Hofmatt & tram \\
983964446 & 8578185 & Münchenstein, Hofmatt & bus \\
1435404358 & 8500095 & Münchenstein, Hofmatt & tram \\
3858822225 & 8578185 & Münchenstein, Hofmatt & bus \\
\bottomrule
\end{tabular}
\end{table}

\FloatBarrier

\subsection{Étape 2 : Correspondance par référence locale dans un rayon de 50 mètres}
Cette sous-étape recherche, pour chaque entrée ATLAS non appariée, un nœud OSM situé à moins de 50 mètres dont la balise \texttt{local\_ref} correspond exactement à la \texttt{designation} de l’entrée ATLAS (en ignorant la casse).

À Zürich HB, dans ATLAS, la \texttt{UIC\_ref} est égale à 8503000 pour tous les arrêts, tandis que dans OSM, certains arrêts ont une \texttt{UIC\_ref} de 8516144. 

\begin{figure}[h]
    \centering
    \includegraphics[height=6cm]{../figures/correspondances/distance_2.png}
    \caption[Correspondances distance étape 2]{Correspondances à Genève-Eaux-Vives-Gare grâce à l'étape 2.}
    \label{fig:distance_2}
\end{figure}

Cette méthode nous a permis de réaliser 129 correspondances supplémentaires.

\subsection{Étape 3 : Correspondance basée sur la proximité avec critères relatifs}  
Pour les entrées toujours non appariées, tous les nœuds OSM situés à moins de 50 mètres sont examinés :  
\begin{itemize}  
    \item a) Si un seul nœud OSM se trouve dans ce rayon, il est apparié à l’entrée ATLAS.  
    \item Si plusieurs nœuds OSM sont présents, l’appariement est effectué avec le nœud le plus proche uniquement si :  
    \begin{enumerate}  
        \item b) Le deuxième nœud le plus proche est à au moins 10 mètres.  
        \item La distance au deuxième nœud le plus proche est au moins 4 fois supérieure à celle du nœud le plus proche.  
    \end{enumerate}  
\end{itemize}  
Nous avons réussi à établir 2\,012 correspondances avec l'option a) et 1\,136 correspondances avec l'option b).
Cette méthode est utile pour les cas où il y a des nœuds isolés, comme des télésièges.  

\begin{figure}[h] 
    \centering
    \includegraphics[width=0.7\textwidth]{../figures/correspondances/distance_3a.png}
    \caption[Correspondance par distance – étape 3a]{Correspondance par distance étape 3 : exemple d'un arrêt isolé où un seul candidat OSM est trouvé dans le rayon de 50 mètres.}
    \label{fig:distance_stage3}
\end{figure} 
\begin{figure}[h] 
    \centering
    \includegraphics[width=0.7\textwidth]{../figures/correspondances/distance_3b.png}
    \caption[Correspondance par distance – étape 3b]{Correspondance par distance étape 3b.}
    \label{fig:distance_stage3b}
\end{figure} 

\begin{tcolorbox}[colback=white, colframe=black!80, arc=3mm, boxrule=1pt]
\textbf{Note :} La méthode de correspondance par routes, qui est la prochaine étape logique dans notre processus séquentiel, est détaillée dans le chapitre suivant en raison de sa complexité.
\end{tcolorbox}

\section{Consolidation Post-traitement}

Après l'exécution des méthodes principales de correspondance (exacte, par nom, par distance et par routes), le système effectue une consolidation post-traitement. Cette étape, appelée "consolidation unique par UIC", examine les entrées ATLAS restantes non appariées et recherche des nœuds OSM disponibles partageant le même identifiant UIC.

Cette consolidation est particulièrement utile dans les cas où :
\begin{itemize}
    \item Des nœuds OSM étaient temporairement indisponibles lors de la correspondance exacte initiale
    \item Des conflits de priorité ont empêché certaines correspondances exactes évidentes
    \item Des entrées ont été filtrées lors des étapes précédentes mais redeviennent candidates valides
\end{itemize}

Le processus fonctionne de manière conservative : il ne crée une correspondance que si exactement un nœud OSM disponible correspond à l'identifiant UIC de l'entrée ATLAS, garantissant ainsi une fiabilité maximale.

Cette étape de consolidation a permis d'identifier 883 correspondances exactes supplémentaires.

\section{Propagation des Duplicatas}

Une dernière étape consiste à propager les correspondances trouvées aux entrées ATLAS dupliquées. Lorsque plusieurs entrées ATLAS partagent les mêmes caractéristiques (numéro et désignation), et qu'une correspondance a été établie pour l'une d'entre elles, cette correspondance est étendue aux autres entrées du groupe dupliqué.

Cette propagation permet d'assurer la cohérence des correspondances et d'améliorer le taux de couverture global, particulièrement dans les grandes gares où plusieurs entrées ATLAS peuvent représenter des aspects différents d'un même arrêt physique.

La propagation des duplicatas a permis d'ajouter 66 correspondances supplémentaires.

\section{Correspondance Manuelle}

Enfin, si les entrées n'ont pas été appariées par les méthodes précédentes, le système applique les correspondances manuelles définies préalablement par les utilisateurs et stockées de manière persistante dans la base de données. 

Les correspondances manuelles sont faites depuis l'application web comme on le verra plus tard.

\section{Résultats actuels}

Sur les \textbf{54\,880} arrêts ATLAS considérés (\texttt{BOARDING\_PLATFORM}), nous obtenons \textbf{48\,213} correspondances, comptées \emph{en paires ATLAS--OSM} (chaque couple \textit{sloid ATLAS}--\textit{nœud OSM}).

Dans cette convention «\ par paires\ », la couverture atteint \textbf{87,9\%} (\(48\,213/54\,880\)). Après déduplication par \textit{sloid} ATLAS, on compte \textbf{46\,611} arrêts appariés (\(46\,611/54\,880 \approx 85,0\%\)).

\bigskip
\begin{tabular}{@{}l r@{\hspace{2em}}r@{}}
\toprule
\textbf{Méthode} & \textbf{Nombre} & \textbf{\%} \\
\midrule
\textcolor{blue}{\textbf{Correspondances exactes}} & \textbf{21\,124} &  \\
\textcolor{purple}{\textbf{Correspondances par nom}} & \textbf{535} &  \\
\textcolor{orange}{\textbf{Correspondances par distance}} & \textbf{18\,661} &  \\
\quad $\bullet$ Étape 1 (groupe-proximité) & 15\,384 &  \\
\quad $\bullet$ Étape 2 (référence locale) & 129 &  \\
\quad $\bullet$ Étape 3a (candidat unique) & 2\,012 &  \\
\quad $\bullet$ Étape 3b (ratio de distance) & 1\,136 &  \\
\textcolor{red}{\textbf{Correspondances par routes}} & \textbf{6\,944} &  \\
\textcolor{darkgray}{\textbf{Consolidation post-traitement}} & \textbf{883} &  \\
\textcolor{darkgray}{\textbf{Propagation des duplicatas}} & \textbf{66} &  \\
\midrule
\textbf{Total des correspondances (paires)} & \textbf{48\,213} &  \\
\textbf{Total des correspondances (arrêts ATLAS distincts)} & \textbf{46\,611} &  \\
\bottomrule
\end{tabular}
\bigskip

Les pourcentages par \emph{méthode} indiquent la part des \textbf{48\,213} correspondances (paires) attribuée à chaque étape et totalisent \(100\%\). La ligne «\ Total des correspondances (paires)\ » renvoie la couverture calculée \emph{par paires} sur \textbf{54\,880} arrêts ATLAS, et le «\ Total des correspondances (arrêts ATLAS distincts)\ » est obtenu en retirant les \textbf{8\,269} entrées non appariées.

Après ces étapes, 8\,269 entrées ATLAS restent non appariées, et 19\,107 nœuds OSM restent inutilisés. Parmi ces nœuds OSM inutilisés, 14\,003 sont associés à au moins une route, 14\,906 possèdent une référence UIC, et 876 ont une référence locale (\texttt{local\_ref}).

Parmi les entrées ATLAS non appariées, 3\,856 n'ont aucun nœud OSM dans un rayon de 50 mètres.


