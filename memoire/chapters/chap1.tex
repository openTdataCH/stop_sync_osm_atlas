% !TeX spellcheck = fr_FR
\chapter{Chapitre 1 : Données ATLAS}

Les données ATLAS, au cœur de cette étude, proviennent de la plateforme Open Transport Data Swiss [\ref{ref:open_transport_data_swiss}]. Cette ressource centralise les données des transports publics en Suisse, offrant une base précieuse pour l’analyse et le développement d’applications.\\
\textit{Sauf indication contraire, toutes les statistiques et cartes de ce chapitre sont calculées sur le snapshot de données du 11 août 2025. À partir de cette version, un filtre géographique (boîte WGS84 approximant la Suisse) est appliqué en plus du code pays UIC=85, afin d'exclure les points situés hors de Suisse malgré un préfixe UIC suisse.}

\medskip
\noindent\textbf{Reproductibilité.} Les figures et statistiques de ce chapitre ont été produites par des scripts reproductibles disponibles sous \texttt{memoire/scripts\_used/} dans le dépôt Git du projet.

\section{Arrêts}

L'analyse des arrêts s'appuie sur le jeu de données \texttt{traffic-points-actual-date} [\ref{ref:traffic_points_actual_date}], qui recense les arrêts de transport public en Suisse avec des informations sur leur localisation et leurs caractéristiques. Ces points peuvent être visualisés sur une carte interactive via l’application web https://atlas.app.sbb.ch/ [\ref{ref:atlas_app_sbb}].

Nous nous concentrons ici sur deux colonnes principales : le \texttt{number} et la \texttt{designation} de chaque arrêt.

Le numéro d’un arrêt correspond à la référence UIC (Union Internationale des Chemins de fer), un standard international permettant d’identifier les lieux de transport public. Les deux premiers chiffres représentent le code du pays ; la Suisse, par exemple, utilise le code 85[\ref{ref:wikipedia_uic_codes}]. Ainsi, un numéro UIC comme \texttt{ 8502034} désigne un arrêt spécifique du réseau suisse.

La colonne \texttt{designation} fait référence à une identification locale : une valeur de \texttt{3} peut, par exemple, indiquer que l’arrêt correspond à la plateforme 3 d’une gare.

Enfin, les données incluent également des informations sur l’opérateur responsable de chaque arrêt, un élément potentiellement utile pour établir des correspondances avec d’autres jeux de données.

Le jeu de données distingue deux types de \texttt{trafficPointElementType} : \texttt{BOARDING\_AREA} et \texttt{BOARDING\_PLATFORM}. Notre analyse se limite aux \texttt{BOARDING\_PLATFORM}, car les \texttt{BOARDING\_AREA} ne disposent pas de coordonnées géographiques. Pour extraire ces informations, nous avons développé un script Python, \texttt{get\_atlas\_data.py}. Extrait simplifié du chargement/filtrage:

\begin{codebox}[language=Python]{Extrait — \texttt{get\_atlas\_stops}}
def get_atlas_stops(output_path, download_url):
    response = requests.get(download_url)
    response.raise_for_status()
    with zipfile.ZipFile(io.BytesIO(response.content)) as z:
        csv_filename = z.namelist()[0]
        with z.open(csv_filename) as f:
            df = pd.read_csv(f, sep=';')
            # Suisse (UIC pays = 85) et coordonnees valides (WGS84)
            df = df[df['uicCountryCode'] == 85]
            df = df.dropna(subset=['wgs84North', 'wgs84East'])
            # Filtre géographique: boîte WGS84 Suisse (lat \in [45.4, 47.9], lon \in [5.7, 10.7])
            df['wgs84North'] = pd.to_numeric(df['wgs84North'], errors='coerce')
            df['wgs84East'] = pd.to_numeric(df['wgs84East'], errors='coerce')
            df = df[df['wgs84North'].between(45.4, 47.9) & df['wgs84East'].between(5.7, 10.7)]
            # Comptage des quais
            boarding_platforms = df[df['trafficPointElementType'] == 'BOARDING_PLATFORM']
            df.to_csv(output_path, sep=';', index=False)
\end{codebox}



\paragraph{Statistiques ATLAS.} Sur l'instantané analysé (après filtre UIC=85 et boîte CH) :
\begin{itemize}
  \item \textbf{Lignes avec coordonnées}: \textit{56\,515}.
  \item \textbf{\texttt{BOARDING\_PLATFORM}}: \textit{55\,823}.
  \item \textbf{UIC distincts (\texttt{number})}: \textit{27\,228}.
  \item \textbf{\texttt{designation} non vides}: \textit{12\,144} (\textit{541} valeurs distinctes).
  \item \textbf{\texttt{designation} manquantes}: \textit{44\,371}, dont \textit{4\,430} cas où l'unique entrée du \texttt{number} est sans désignation.
  \item \textbf{Entrées identifiables par (\texttt{number}, \texttt{designation}) seul}: \textit{10\,607}.
  \item \textbf{Total identifiables par \texttt{number} + (\texttt{designation} ou unicité du \texttt{number})}: \textit{16\,014}.
\end{itemize}

\begin{figure}[H]
  \centering
  \includegraphics[width=.76\linewidth]{figures/plots/atlas_points_switzerland.png}
  \caption[ATLAS: distribution nationale]{ATLAS: distribution nationale (WGS84).}
  \label{fig:atlas_ch_points}
\end{figure}

\begin{figure}[H]
  \centering
  \includegraphics[width=.85\linewidth]{figures/plots/atlas_designation_operators.png}
  \caption[Désignations et opérateurs ATLAS]{À gauche: désignations de plateforme les plus fréquentes (hors valeurs manquantes). À droite: principales organisations (abrégés) déclarées.}
  \label{fig:atlas_distribs}
\end{figure}

\noindent\textit{Remarque.} Les valeurs manquantes de \texttt{designation} (très nombreuses) sont exclues du classement pour éviter un effet disproportionné. Comme indiqué plus haut, \(44\,370\) entrées n'ont pas de \texttt{designation}.

Concernant les coordonnées, le fichier fournit deux systèmes : le système de référence suisse LV95 et le système de référence global WGS84. Étant donné que les données d'OpenStreetMap (OSM) utilisent les coordonnées WGS84, nous nous concentrerons uniquement sur cet ensemble de coordonnées pour le moment.

\section{GTFS}

Le General Transit Feed Specification (GTFS) est un format d’échange numérique initié par Google pour standardiser les horaires des transports publics et leurs informations géographiques, telles que la localisation des arrêts. En Suisse, ces données sont publiées sur la plateforme OpenTransportDataSuisse. Elles servent à développer des applications pratiques, comme les outils de consultation d’horaires ou de planification de trajets.

Bien que notre projet se focalise actuellement sur la synchronisation des arrêts, les données GTFS relatives aux trajets suscitent également notre intérêt. Elles pourraient faciliter la correspondance entre les arrêts ATLAS et ceux d’OpenStreetMap, en exploitant les informations sur les itinéraires présentes dans les deux ensembles de données. Parmi les  fichiers GTFS, quatre retiennent notre attention : \texttt{stops.txt}, \texttt{stop\_times.txt}, \texttt{routes.txt} et \texttt{trips.txt}.

\subsection{Description des fichiers}

Les fichiers GTFS suivants sont cruciaux pour notre analyse :

\begin{itemize}
    \item \textbf{\texttt{stops.txt}} : Ce fichier répertorie les arrêts avec leurs coordonnées géographiques et d’autres attributs. 
    Un extrait est présenté dans le tableau \ref{tab:stops}.
    \item \textbf{\texttt{routes.txt}} : Il décrit les lignes de transport, avec des informations comme le nom court, le nom long, et le type de transport. 
    Voir le tableau \ref{tab:routes}.
    \item \textbf{\texttt{trips.txt}} : Ce fichier associe les trajets aux lignes et aux services.
    Un exemple est donné dans le tableau \ref{tab:trips}.
    \item \textbf{\texttt{stop\_times.txt}} : Il contient les horaires d’arrivée et de départ pour chaque arrêt d’un trajet.
    Voir le tableau \ref{tab:stop_times}.
    \begin{table}[H]
    \caption{Extrait du fichier \texttt{stops.txt}}
    \label{tab:stops}
    \centering
    \begin{tabular}{l l r r l l}
    \toprule
    stop\_id & stop\_name & stop\_lat & stop\_lon  & parent\_station \\
    \midrule
    1101064 & Malpensa Aeroporto, terminal 1 & 45.6272 & 8.7111 & \\
    8000339 & Weissenhorn Eschach & 48.3010 & 10.1351  & \\
    8000709:0:2 & Neckarsulm Mitte & 49.1935 & 9.2229  & \\
    8000778 & Asselheim (D) & 49.5762 & 8.1616  & \\
    8000781 & Grünstadt-Nord & 49.5734 & 8.1708 & \\
    8000988 & Witzighausen & 48.3174 & 10.0978 & \\
    8002015 & Nördlingen & 48.8508 & 10.4979 & 8002015P \\
    8002015:0:4 & Nördlingen & 48.8509 & 10.4979 & 8002015P \\
    \bottomrule
    \end{tabular}
    \end{table}

    \begin{table}[H]
    \caption{Extrait du fichier \texttt{routes.txt}}
    \label{tab:routes}
    \centering
    \begin{tabular}{l l l l l l}
    \toprule
    route\_id & agency\_id & route\_short\_name & route\_desc & route\_type \\
    \midrule
    91-10-A-j22-1 & 37 & 10 & T & 900 \\
    91-10-B-j22-1 & 78 & S10 & S & 109 \\
    91-10-C-j22-1 & 11 & S10 & S & 109 \\
    91-10-E-j22-1 & 65 & S10 & S & 109 \\
    91-10-F-j22-1 & 11 & RE10 & RE & 106 \\
    91-10-G-j22-1 & 11 & SN10 & SN & 109 \\
    91-10-j22-1 & 3849 & 10 & T & 900 \\
    91-10-Y-j22-1 & 82 & IR & IR & 103 \\
    \bottomrule
    \end{tabular}
    \end{table}

    \begin{table}[H]
    \caption{Extrait du fichier \texttt{trips.txt}}
    \label{tab:trips}
    \centering
    \begin{tabular}{l l l l l l l l l}
    \toprule
    route\_id & trip\_id & trip\_short\_name & direction\_id  \\
    \midrule
    91-8-H-j25-1  & 994.TA.91-8-H-j25-1.59.R  & 6278 & 1 \\
    91-8-H-j25-1  & 995.TA.91-8-H-j25-1.59.R  & 2978 & 1 \\
    91-8-H-j25-1  & 996.TA.91-8-H-j25-1.59.R & 2787 & 1 \\
    91-8-H-j25-1  & 997.TA.91-8-H-j25-1.59.R  & 4879 & 1 \\
    91-8-H-j25-1  & 998.TA.91-8-H-j25-1.59.R  & 10407 & 1 \\
    91-8-H-j25-1  & 999.TA.91-8-H-j25-1.59.R  & & 1 & \\
    \bottomrule
    \end{tabular}
    \end{table}

    \begin{table}[H]
    \caption{Extrait du fichier \texttt{stop\_times.txt}}
    \label{tab:stop_times}
    \centering
    \begin{tabular}{l l l l r l l}
    \toprule
    trip\_id & arrival\_time & departure\_time & stop\_id & stop\_sequence \\
    \midrule
    1.TA.1-9-j17-1.1.H & 05:25:00 & 05:25:00 & 8502034:0:2 & 1 \\
    1.TA.1-9-j17-1.1.H & 05:28:00 & 05:29:00 & 8502033:0:2 & 2 \\
    1.TA.1-9-j17-1.1.H & 05:33:00 & 05:33:00 & 8502032:0:1 & 3 \\
    1.TA.1-9-j17-1.1.H & 05:36:00 & 05:36:00 & 8502031:0:1 & 4 \\
    1.TA.1-9-j17-1.1.H & 05:42:00 & 05:42:00 & 8502030:0:2 & 5 \\
    1.TA.1-9-j17-1.1.H & 05:50:00 & 05:50:00 & 8502119:0:7 & 6 \\
    2.TA.1-9-j17-1.2.H & 05:53:00 & 05:53:00 & 8502034:0:1 & 1 \\
    2.TA.1-9-j17-1.2.H & 05:57:00 & 05:58:00 & 8502033:0:2 & 2 \\
    \bottomrule
    \end{tabular}
    \end{table}
\end{itemize}


\subsection{Clés d’identification, normalisation et jointure}

Dans le script \texttt{get\_atlas\_data.py}, nous construisons un jeu intégré qui associe chaque arrêt aux couples \texttt{(route\_id, direction\_id)} desservis et aux noms de lignes, puis relions ces arrêts aux SLOIDs ATLAS. La jointure exploite \texttt{stop\_times.txt} pour relier les arrêts aux trajets via \texttt{trip\_id}, puis \texttt{trips.txt} et \texttt{routes.txt} pour relier ces trajets aux lignes via \texttt{route\_id}. Nous dédupliquons ensuite les paires route–direction par arrêt et ajoutons le nom de ligne.

\medskip
\noindent\textbf{Clé de correspondance stop\_id GTFS \(\to\) SLOID ATLAS.} Nous associons \texttt{stop\_id} (GTFS) aux SLOIDs ATLAS via \texttt{number} (UIC) et \texttt{designation} (référence locale de quai), avec normalisation minimale. Les règles \emph{implémentées dans le code} sont:
\begin{itemize}
  \item \textbf{Structure de \texttt{stop\_id}}: \texttt{uic\_number:0:local\_ref}. Exemple: \texttt{8516155:0:1}.
  \item \textbf{Strict}: associer si \(\texttt{uic\_number} = \texttt{number}\) et \(\texttt{normalized\_local\_ref} = \texttt{designation}\).
  \item \textbf{Fallback 1} (si non associé strictement): si le \texttt{number} côté ATLAS n’a qu’une seule ligne, utiliser son \texttt{sloid}.
  \item \textbf{Fallback 2} (sinon): si la dernière composante du \texttt{sloid} (après les \texttt{:}) est égale à \texttt{normalized\_local\_ref}, utiliser ce \texttt{sloid}.
  \item \textbf{Normalisation}: les références locales \enquote{10000/10001} sont ramenées à \enquote{1/2} lorsqu’elles codent des côtés/plates-formes.
\end{itemize}

La mise en correspondance entre la colonne \texttt{stop\_id} du fichier \texttt{stops.txt} (GTFS) et l’identifiant \texttt{sloid} d’ATLAS présente des défis significatifs. Premièrement, il n’existe pas de lien direct entre ces deux identifiants. Deuxièmement, les coordonnées géographiques des arrêts diffèrent entre les deux ensembles de données.

\textbf{Exemple 1 : "Lancy-Pont-Rouge":}
\newline
Considérons la gare "Lancy-Pont-Rouge", opérée par les CFF. Dans le fichier \texttt{stops.txt} de GTFS, les données sont les suivantes :

\begin{table}[H]
\caption{Extrait de \texttt{stops.txt} pour "Lancy-Pont-Rouge"}
\label{tab:stops_lancy_2}
\centering
\begin{tabular}{l l r r l l}
\toprule
\texttt{stop\_id} & \texttt{stop\_name} & \texttt{stop\_lat} & \texttt{stop\_lon} & \texttt{parent\_station} \\
\midrule
8516155:0:1 & Lancy-Pont-Rouge & 46.18596197 & 6.12483039 & Parent8516155 \\
8516155:0:2 & Lancy-Pont-Rouge & 46.18595575 & 6.12495615 & Parent8516155 \\
\bottomrule
\end{tabular}
\end{table}

Dans le fichier \texttt{traffic-points-actual-data}, on trouve :

\begin{table}[H]
\caption{Extrait de \texttt{traffic-points-actual-data} pour "Lancy-Pont-Rouge"}
\label{tab:traffic_lancy_2}
\centering
\begin{tabular}{l l l l r r l}
\toprule
\texttt{sloid} & \texttt{number} & \texttt{des.} & \texttt{wgs84East} & \texttt{wgs84North} & \texttt{designationOfficial} \\
\midrule
...:16155:1:1 & 8516155 & 1  & 6.12483137 & 46.18596333 & Lancy-Pont-Rouge \\
...:16155:1:2 & 8516155 & 2  & 6.12495213 & 46.18595284 & Lancy-Pont-Rouge \\
\bottomrule
\end{tabular}
\end{table}

Ici, le format de \texttt{stop\_id} dans GTFS est \texttt{uic\_number:0:local\_ref}, où \texttt{uic\_number} correspond à la colonne \texttt{number} dans ATLAS (8516155), et \texttt{local\_ref} à \texttt{designation} (1 ou 2). Cela permet une correspondance, bien que les coordonnées géographiques divergent légèrement.

\textbf{Exemple 2 : "Lausanne Bourdonnette":}
\newline
Prenons un deuxième exemple avec "Lausanne Bourdonnette". Dans \texttt{stops.txt} :
\begin{table}[H]
\caption{Extrait de \texttt{stops.txt} pour "Lausanne Bourdonnette"}
\label{tab:stops_bourdonnette_2}
\centering
\begin{tabular}{l l r r}
\toprule
\texttt{stop\_id} & \texttt{stop\_name} & \texttt{stop\_lat} & \texttt{stop\_lon} \\
\midrule
8501210:0:10000 & Lausanne, Bourdonnette & 46.52342565 & 6.59074161 \\
8501210:0:10001 & Lausanne, Bourdonnette & 46.52329585 & 6.58987025 \\
8501210:0:A & Lausanne, Bourdonnette & 46.52326494 & 6.58980736 \\
8501210:0:B & Lausanne, Bourdonnette & 46.52318459 & 6.58978940 \\
8501210:0:C & Lausanne, Bourdonnette & 46.52272720 & 6.58913363 \\
8501210:0:D & Lausanne, Bourdonnette & 46.52338238 & 6.59138840 \\
\bottomrule
\end{tabular}
\end{table}

Et dans \texttt{traffic-points-actual-data} :

\begin{table}[H]
\caption{Extrait de \texttt{traffic-points-actual-data} pour "Lausanne Bourdonnette"}
\label{tab:traffic_bourdonnette_2}
\centering
\begin{tabular}{l l l l r r l}
\toprule
\texttt{sloid} & \texttt{number} & \texttt{des.
}  & \texttt{wgs84East} & \texttt{wgs84North} & \texttt{designationOfficial} \\
\midrule
...:1210:0:1600 & 8501210 &  & 6.59074107 & 46.52342597 & Lausanne, Bourdonnette \\
...:1210:0:1610 & 8501210 &  & 6.58986994 & 46.52329351 & Lausanne, Bourdonnette \\
...:1210:0:1616 & 8501210 & B & 6.58979344 & 46.52318499 & Lausanne, Bourdonnette \\
...:1210:0:2597 & 8501210 & D & 6.59138793 & 46.52338108 & Lausanne, Bourdonnette \\
...:1210:0:2542 & 8501210 & C & 6.58913042 & 46.52272550 & Lausanne, Bourdonnette \\
\bottomrule
\end{tabular}
\end{table}

Dans ce cas, les désignations dans GTFS incluent "A", "B", "C", "D", ainsi que des références numériques comme "10000" et "10001", mais dans ATLAS,  "A" n’a pas d’équivalent direct, et les références numériques ne sont pas assignées (lignes avec \texttt{designation} vide). Les coordonnées géographiques diffèrent également.

\subsection{Résultats}
Nous obtenons un jeu intégré listant, par \texttt{sloid}, les couples \texttt{(route\_id, direction\_id)}. Sur notre jeu:
\begin{itemize}
  \item \textbf{SLOIDs couverts par GTFS}: \textit{32\,248}.
  \item \textbf{Médiane des lignes par SLOID (GTFS)}: \textit{2}.
\end{itemize}

\begin{figure}[h]
  \centering
  \includegraphics[width=.7\linewidth]{figures/plots/gtfs_routes_per_sloid.png}
  \caption[GTFS: lignes par SLOID]{GTFS: distribution du nombre de lignes par SLOID.}
  \label{fig:gtfs_lines_per_sloid}
\end{figure}

\begin{figure}[h]
  \centering
  \includegraphics[width=.76\linewidth]{figures/plots/gtfs_points_switzerland.png}
  \caption[Arrêts GTFS (Suisse)]{Arrêts GTFS (Suisse): \(47\,846\) arrêts suisses détectés.}
  \label{fig:gtfs_ch_points}
\end{figure}




\vspace{.2em}

\section{HRDF}
Le \textit{HAFAS Raw Data Format} structure des horaires exhaustifs. Deux fichiers sont clés: \texttt{BHFART} (SLOIDs gare/quai) et \texttt{GLEISE\_WGS/LV95} (infrastructure et coordonnées de quai).

\subsection{Comment nous le lisons}
Notre extraction s’effectue en deux passes efficaces et ciblées:
\begin{enumerate}
  \item \textbf{GLEISE\_LV95 \(\to\) paires (UIC, \#ref) par \texttt{sloid}.} Nous parcourons \texttt{GLEISE\_LV95} pour associer chaque \texttt{sloid} de quai au \texttt{UIC} de gare et au numéro de référence (\#ref) qui identifie le quai.
  \item \textbf{FPLAN \(\to\) directions.} Pour ces \((\text{UIC}, \#\text{ref})\) cibles seulement, nous analysons \texttt{FPLAN} afin d’extraire, par voyage, le premier et le dernier arrêt. En reliant ces arrêts à leurs noms (via \texttt{BAHNHOF}), nous formons des chaînes directionnelles \textit{UIC} (\enquote{8501008 \(\to\) 8501120}).
\end{enumerate}



\subsection{Informations exploitées}
\begin{itemize}
  \item \textbf{SLOID de quai} et position (WGS84);
  \item \textbf{Chaînes directionnelles} \textit{nom} (\textit{Genève} \(\to\) \textit{Lausanne}) et \textit{UIC} (\(8501008 \to 8501120\)).
\end{itemize}

\subsection{Statistiques clés}
\noindent\textit{Compte les SLOIDs de quai uniques référencés dans \texttt{GLEISE\_LV95}.}
\begin{cmdbox}
$ grep -o "g A ch:1:sloid:[^[:space:]]\\+" data/raw/GLEISE_LV95 | \\
  sed 's/^g A //' | sort -u | wc -l
30935
\end{cmdbox}

\noindent\textit{Compte les SLOIDs de quai uniques dans \texttt{BHFART} (entrées \enquote{G a}).}
\begin{cmdbox}
$ grep -o "G a ch:1:sloid:[^[:space:]]\\+" data/raw/BHFART | sed 's/^G a //' | sort -u | wc -l
30935
\end{cmdbox}

\noindent\textit{Compte les SLOIDs de gare uniques dans \texttt{BHFART} (entrées \enquote{G A}).}
\begin{cmdbox}
$ grep -o "G A ch:1:sloid:[^[:space:]]\\+" data/raw/BHFART | sed 's/^G A //' | sort -u | wc -l
31913
\end{cmdbox}


\noindent
Après extraction des informations de direction : \textbf{\(28\,757\)} SLOIDs couverts ; médiane des directions (noms) par SLOID : \textbf{4}. Ces chaînes \enquote{nom} et \enquote{UIC} enrichissent les correspondances quand un identifiant de direction explicite n’est pas disponible côté OSM.

\begin{figure}[H]
  \centering
  \includegraphics[width=.76\linewidth]{figures/plots/hrdf_quays_switzerland.png}
  \caption[Quais HRDF – Suisse]{Quais HRDF (\texttt{GLEISE\_WGS}) – Suisse.}
\end{figure}


\section{Comparaison GTFS et HRDF (couverture SLOID)}

% 1-plot figure: ATLAS Genève (quais)
\begin{figure}[H]
  \centering
  \includegraphics[width=.7\linewidth]{figures/plots/atlas_points_geneva.png}
  \caption[ATLAS – Genève]{ATLAS – plateformes d'embarquement, zoom Genève.}
  \label{fig:atlas_geneva}
\end{figure}

% 2-plot figure: GTFS vs HRDF (points bruts) – Genève
\begin{figure}[H]
  \centering
  \begin{minipage}[t]{0.49\linewidth}
    \centering
    \includegraphics[width=\linewidth]{figures/plots/gtfs_points_geneva.png}
    \vspace{0.2em}
    \small GTFS – arrêts (zoom Genève)
  \end{minipage}\hfill
  \begin{minipage}[t]{0.49\linewidth}
    \centering
    \includegraphics[width=\linewidth]{figures/plots/hrdf_quays_geneva.png}
    \vspace{0.2em}
    \small HRDF – quais (zoom Genève)
  \end{minipage}
  \caption[Genève: GTFS vs HRDF (points bruts)]{Genève: comparaison des points bruts \textbf{GTFS} (gauche) et \textbf{HRDF} (droite).}
  \label{fig:geneva_gtfs_hrdf_raw}
\end{figure}

% 2-plot figure: SLOIDs appariés (GTFS vs HRDF) – Genève
\begin{figure}[H]
  \centering
  \includegraphics[width=.95\linewidth]{figures/plots/geneva_matched_sloids_gtfs_hrdf.png}
  \caption[Genève: SLOIDs appariés (GTFS vs HRDF)]{Genève: SLOIDs ATLAS couverts par les jeux intégrés \textbf{GTFS} (gauche) et \textbf{HRDF} (droite).}
  \label{fig:geneva_matched_sloids}
\end{figure}

\begin{itemize}
  \item GTFS: \textit{\(34\,766\)} SLOIDs\,; HRDF: \textit{\(28\,757\)} SLOIDs.
  \item Intersection: \textit{\(14\,238\)}\,; GTFS seulement: \textit{\(20\,528\)}\,; HRDF seulement: \textit{\(14\,519\)}.
\end{itemize}