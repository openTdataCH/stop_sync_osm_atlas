% !TeX spellcheck = fr_FR
\chapter{Chapitre 11: Évaluation de la sécurité de l'application}\label{chap:security}

\noindent Ce chapitre complète le \textbf{Chap.~\ref{chap:auth}} en évaluant les défenses mises en place, en illustrant les scénarios d'attaque plausibles et en discutant des améliorations futures. L'objectif est comprendre comment et pourquoi les contrôles bloquent les attaques, et où on pourrait renforcer la posture.

\section{Périmètre et surface d'attaque}
\noindent Pour rendre la lecture concrète, on part d'une \emph{carte des routes} – celles que l'utilisateur peut toucher. Nous indiquons les garde-fous (authentification, limites de débit) directement à côté de chaque groupe.

\paragraph{Pages (UI)} \texttt{/}, \texttt{/map\_snapshot}, \texttt{/problems}, \texttt{/persistent\_data}, \texttt{/reports} — vues HTML publiques qui appellent l'API ci-dessous.

\paragraph{Authentification} (limitées par IP)
\begin{itemize}
  \item \texttt{/auth/register} \textit{GET,POST} — \textbf{5/min}; CAPTCHA Turnstile
  \item \texttt{/auth/login} \textit{GET,POST} — \textbf{10/min}; CAPTCHA + verrouillage progressif par compte
  \item \texttt{/auth/2fa} \textit{GET,POST} — \textbf{15/min}
  \item \texttt{/auth/logout} \textit{POST} — \textbf{auth requis}
  \item \texttt{/auth/enable\_2fa} \textit{GET,POST} — \textbf{auth requis}
  \item \texttt{/auth/disable\_2fa} \textit{POST} — \textbf{auth requis}
  \item \texttt{/auth/verify-email/<token>} \textit{GET} — \textbf{30/h}
  \item \texttt{/auth/resend-verification} \textit{GET,POST} — \textbf{5/min} (réponse neutre)
  \item \texttt{/auth/status} \textit{GET}
\end{itemize}

\paragraph{Données} (lecture cartographique)
\begin{itemize}
  \item \texttt{/api/data} \textit{GET} — \textbf{30/min}; filtrage \emph{SARGable} sur la fenêtre \texttt{bbox}, pagination
  \item \texttt{/api/stop\_popup} \textit{GET} — \textbf{120/min}; jointures ciblées, sérialisation compacte
  \item \texttt{/api/route\_stops} \textit{GET} — \textbf{60/min}; requête optimisée + fallback de normalisation
  \item \texttt{/api/operators} \textit{GET} — \textbf{60/min}; \texttt{SELECT DISTINCT} indexé
\end{itemize}

\paragraph{Recherche}
\begin{itemize}
  \item \texttt{/api/search} \textit{GET} — \textbf{60/min}; colonnes réduites via \texttt{optimize\_query\_for\_endpoint}
  \item \texttt{/api/top\_matches} \textit{GET} — \textbf{60/min}; tri sur colonne indexée \texttt{distance\_m}
  \item \texttt{/api/random\_stop} \textit{GET} — \textbf{30/min}; \emph{sampling} par plages d'ID (pas de \texttt{ORDER BY RAND()})
  \item \texttt{/api/stop\_by\_id} \textit{GET} — \textbf{60/min}
  \item \texttt{/api/manual\_match} \textit{POST} — \textbf{30/min}; enregistre un appariement manuel
\end{itemize}

\paragraph{Statistiques et rapports}
\begin{itemize}
  \item \texttt{/api/global\_stats} \textit{GET} — \textbf{30/min}; cache LRU en mémoire
  \item \texttt{/api/generate\_report} \textit{GET} — \textbf{auth requis}, \textbf{20/jour}; génération PDF/CSV côté serveur
\end{itemize}

\paragraph{Problèmes et persistance}
\begin{itemize}
  \item \texttt{/api/problems}, \texttt{/api/problems/stats} \textit{GET} — \textbf{120/min}
  \item \texttt{/api/save\_solution}, \texttt{/api/make\_solution\_persistent} \textit{POST} — \textbf{30/min}, \textbf{auth requis}
  \item \texttt{/api/save\_note/atlas}, \texttt{/api/save\_note/osm}, \texttt{/api/make\_note\_persistent/<type>} \textit{POST} — \textbf{60/min}, \textbf{auth requis}
  \item \texttt{/api/check\_persistent\_solution}, \texttt{/api/check\_persistent\_note/...} \textit{GET} — \textbf{120/min}
  \item \texttt{/api/persistent\_data}, \texttt{/api/non\_persistent\_data} \textit{GET} — \textbf{60/min}, \textbf{auth requis}
  \item \texttt{/api/persistent\_data/<id>} \textit{DELETE} — \textbf{30/min}, \textbf{auth+admin}
  \item \texttt{/api/make\_non\_persistent/<id>}, \texttt{/api/clear\_all\_persistent}, \texttt{/api/clear\_all\_non-persistent}, \texttt{/api/make\_all\_persistent} \textit{POST} — \textbf{10/h}, \textbf{auth requis} (admin si destructif)
\end{itemize}

\noindent Types d'attaques considérés: \textbf{bourrage d'identifiants}, \textbf{force brute}, \textbf{contournement 2FA}, \textbf{énumération d'email}, \textbf{CSRF}, \textbf{abus de liens signés} et \textbf{déni de service (DoS)}.

\section{Contrôles en place (aperçu technique)}

\subsection*{Mots de passe et stockage}
\noindent Les mots de passe sont hachés avec Argon2id via \texttt{argon2-cffi}, avec paramètres adaptés à la mémoire. Vérification sûre:
\begin{codebox}[language=Python]{Vérifier un mot de passe}
from argon2 import PasswordHasher
ph = PasswordHasher()
ph.verify(user.password_hash, candidate)
\end{codebox}

\subsection*{Sessions et cookies}
\noindent Cookies \texttt{HttpOnly}, \texttt{SameSite=Lax} et \texttt{Secure} activable par variable d'environnement (produit: à forcer). Durée \og remember \fg{} de 14~jours.
\begin{codebox}[language=Python]{Paramètres de session (extrait)}
app.config['SESSION_COOKIE_HTTPONLY'] = True
app.config['SESSION_COOKIE_SAMESITE'] = 'Lax'
app.config['SESSION_COOKIE_SECURE'] = os.getenv('SESSION_COOKIE_SECURE','false').lower()=='true'
\end{codebox}

\subsection*{CSRF et pages de formulaire}
\noindent Les formulaires d'authentification utilisent Flask-WTF/CSRF. Certaines API JSON historiques sont explicitement exemptées (lecture seule), pas les routes d'authentification.

\subsection*{CAPTCHA Turnstile}
\noindent Un CAPTCHA Cloudflare Turnstile protège inscription et connexion; désactivé automatiquement en développement si clés absentes.
\begin{codebox}[language=Python]{Vérifier le CAPTCHA (simplifié)}
secret = os.getenv('TURNSTILE_SECRET_KEY','')
if not secret: return True  # dev
resp = requests.post('https://.../siteverify', data={...})
ok = bool(resp.json().get('success'))
\end{codebox}

\subsection*{Limitation de débit et verrouillage}
\noindent Double barrière: limites par route \emph{et} verrouillage progressif par compte après échecs répétés.
\begin{codebox}[language=Python]{Limites de débit par route}
@limiter.limit("5/minute")   # /auth/register
@limiter.limit("10/minute")  # /auth/login
@limiter.limit("15/minute")  # /auth/2fa
@limiter.limit("30/hour")    # /auth/verify-email/<token>
@limiter.limit("30/minute")  # /api/data
@limiter.limit("120/minute") # /api/stop_popup
@limiter.limit("60/minute")  # /api/search, /api/top_matches, /api/route_stops, /api/operators
@limiter.limit("30/minute")  # /api/random_stop, /api/manual_match
@limiter.limit("30/minute")  # /api/global_stats (avec cache LRU)
@login_required; @limiter.limit("20/day")  # /api/generate_report
\end{codebox}

\begin{codebox}[language=Python]{Verrouillage exponentiel (extrait)}
if not user.verify_password(password):
    user.failed_login_attempts += 1
    if user.failed_login_attempts >= 5:
        lock_minutes = min(60, 2 * user.failed_login_attempts + 5)
        user.locked_until = utcnow() + timedelta(minutes=lock_minutes)
\end{codebox}

\subsection*{2FA TOTP et codes de secours}
\noindent TOTP standard (30~s) via \texttt{pyotp}. Les codes de secours sont \textbf{hachés} avec Argon2 et \textbf{consommés} à l'usage.
\begin{codebox}[language=Python]{Vérifier TOTP ou code de secours}
totp_ok = pyotp.TOTP(user.totp_secret).verify(token, valid_window=1)
backup_ok = user.verify_and_consume_backup_code(token)
\end{codebox}

\subsection*{Journalisation d'audit}
\noindent Chaque tentative ou succès d'action sensible est consignée dans \texttt{auth\_db.auth\_events} et émise en JSON dans les logs:\
\begin{itemize}
  \item \textbf{Événements} : inscription, connexion (succès/échec), verrouillage, 2FA (succès/échec), activation/désactivation 2FA, email vérifié, logout.
  \item \textbf{Contexte} : IP (\texttt{X-Forwarded-For} prioritaire), \texttt{User-Agent}, email tenté (si échec), métadonnées.
  \item \textbf{Exploitation} : requêtes SQL côté \texttt{auth\_db} ou filtrage des logs \texttt{docker compose logs}.
\end{itemize}

\begin{codebox}[language=SQL]{Exemples de requêtes}
-- Connexions échouées pour un email sur 24 h
SELECT occurred_at, ip_address, metadata_json
FROM auth_events
WHERE event_type = 'login_failure'
  AND email_attempted = 'user@example.com'
  AND occurred_at > NOW() - INTERVAL 1 DAY
ORDER BY occurred_at DESC;
\end{codebox}

\begin{codebox}[language=bash]{Filtrer les événements dans les logs du conteneur}
docker compose logs -f app | grep 'auth_event' | cat
\end{codebox}

\section{Scénarios d'attaque et déroulé}

\subsection*{A1 — Bourrage d'identifiants / force brute}
\textbf{Attaque}. Un robot tente des listes \og email+mot de passe \fg{}.

\textbf{Côté serveur}. La route \texttt{/auth/login} est plafonnée à 10/min par IP; les échecs incrémentent un compteur par compte. Après 5 échecs: verrouillage progressif (5, 7, 9, ... minutes jusqu'à 60 min max). Cookies non délivrés tant que la session n'est pas authentifiée.

\textbf{Résultat}. Les rafales sont ralenties par IP (limiteur) et par compte (verrouillage), ce qui rend l'attaque coûteuse et lente. Des détails supplémentaires de calibrage sont discutés ci-dessous.

\begin{figure}[h]
  \centering
  \fbox{\parbox{0.9\textwidth}{\centering Placeholder: \emph{Courbe du temps d'attente cumulé} vs nombre d'essais échoués.}}
  \caption{Verrouillage progressif: coût croissant pour l'attaquant.}
\end{figure}

\subsection*{A2 — Contournement 2FA}
\textbf{Attaque}. L'attaquant dérobe un mot de passe (hameçonnage) et tente de se connecter sans 2FA.

\textbf{Côté serveur}. Si \texttt{is\_totp\_enabled=true}, une étape 2FA obligatoire s'intercale. Un TOTP valide (\textpm 30~s) ou un code de secours non consommé est requis.

\textbf{Résultat}. Sans le second facteur (ou un code de secours), l'intrusion échoue. Les codes de secours étant \textbf{hachés} et \textbf{consommés}, leur ré-utilisation est impossible.

\begin{figure}[h]
  \centering
  \fbox{\parbox{0.9\textwidth}{\centering Placeholder: \emph{Diagramme de séquence} \og login + 2FA \fg{} avec embranchements TOTP/backup.}}
  \caption{Garde 2FA: étape obligatoire après mot de passe.}
\end{figure}

\paragraph{Note}. Le secret TOTP est stocké en clair dans \texttt{auth\_db} (voir \S~\ref{sec:improve}). Cela protège contre les vols de base non, mais facilite la rotation et l'interopérabilité. Un chiffrement applicatif à la volée est recommandé en production.

\subsection*{A3 — Énumération d'email}
\textbf{Attaque}. Tester si un email existe via les messages d'erreur.

\textbf{Côté serveur}. \texttt{/auth/login} répond \og Invalid credentials \fg{} dans tous les cas; \texttt{/auth/resend-verification} répond identiquement qu'un compte existe ou non. \textbf{Exception}: \texttt{/auth/register} indique si l'email existe déjà.

\textbf{Résultat}. L'énumération est évitée sur login/resend, mais possible sur register. Voir \S~\ref{sec:improve} pour uniformiser les messages.

\subsection*{A4 — Abus de liens signés (vérification d'email)}
\textbf{Attaque}. Réutiliser un lien, tenter un bruteforce de jetons ou une inondation d'emails.

\textbf{Côté serveur}. Jetons \texttt{itsdangerous} avec \textbf{sel dédiée} et péremption 48~h; limitation 30/h sur \texttt{/auth/verify-email/...}. Rythme d'envoi côté serveur limité (1 email/min par compte) et limite 5/min sur \texttt{/auth/resend-verification}.

\textbf{Résultat}. Les relectures expirent, les tentatives massives sont ralenties.

\begin{figure}[h]
  \centering
  \fbox{\parbox{0.9\textwidth}{\centering Placeholder: \emph{Timeline} d'un jeton: émission, délai de 48~h, invalidation.}}
  \caption{Cycle de vie d'un jeton de vérification.}
\end{figure}

\subsection*{A5 — CSRF sur routes sensibles}
\textbf{Attaque}. Forcer une action authentifiée via une page tierce.

\textbf{Côté serveur}. Les formulaires d'authentification (POST) sont protégés par CSRF. Les API JSON en lecture restent exemptées.

\textbf{Résultat}. Les tentatives de soumission cachée échouent faute de jeton.

\subsection*{A6 — DoS sur endpoints lourds}
\textbf{Attaque}. Bombarder \texttt{/api/data}, \texttt{/api/stop\_popup} ou \texttt{/api/generate\_report} pour saturer la base/le CPU.

\textbf{Côté serveur}. Nous avons ajouté ou resserré des limites: \texttt{/api/data} \textbf{30/min}, \texttt{/api/stop\_popup} \textbf{120/min}, \texttt{/api/search/top\_matches} \textbf{60/min}, \texttt{/api/global\_stats} \textbf{30/min} (avec \textbf{cache LRU}), \texttt{/api/generate\_report} désormais \textbf{authentifiée} et plafonnée à \textbf{20/jour}. Les requêtes sont \emph{SARGable} (filtres par fenêtre) et paginées pour bornes fortes.

\textbf{Résultat}. Un attaquant doit multiplier les IPs et comptes authentifiés pour maintenir une charge significative; l'impact reste contenu. Les métriques d'audit aident à repérer des rafales anormales.

\section{Calibrage et limites actuelles}
\noindent Les limites en place forment un socle solide mais peuvent être durcies:
\begin{itemize}
  \item \textbf{Limiteur par IP} : efficace mais contournable via réseaux distribués; ajouter une clé composite (IP + email cible) et des \og seaux \fg{} par utilisateur.
  \item \textbf{Verrouillage par compte} : robuste, mais attention au déni de service ciblé (un adversaire peut \og verrouiller \fg{} le compte d'une victime). Des \emph{captcha} et \emph{cooldowns} nuancés aident à mitiger.
  \item \textbf{Remember cookie de 14~jours} : confortable; réduire la durée ou exiger 2FA périodiquement renforce la sécurité.
\end{itemize}

\section{Améliorations et vecteurs non encore couverts}\label{sec:improve}
\noindent Mesures recommandées, de l'immédiat au stratégique:
\begin{itemize}
  \item \textbf{Forcer HTTPS et cookies Secure} partout (\texttt{FORCE\_HTTPS=true}, \texttt{SESSION\_COOKIE\_SECURE=true}).
  \item \textbf{Politiques CSP strictes} (actuellement désactivées) avec listes blanches minimales, ou auto-hébergement des actifs.
  \item \textbf{Unifier les messages d'inscription} pour éviter l'énumération d'emails (réponse neutre: \og si un compte existe déjà, vous recevrez un email \fg{}).
  \item \textbf{Clés de limite composites} : ajouter des limites \emph{par compte} (IP+email) et des \emph{burst tokens} pour lisser.
  \item \textbf{Écriture manuelle}: exiger l'authentification forte pour tout endpoint mutateur (p.\,ex. \texttt{/api/manual\_match}) et journaliser finement.
  \item \textbf{Chiffrement au repos du secret TOTP} (p.~ex. libsodium/\texttt{Fernet} avec rotation de clés) et \textbf{réauthentification} exigée pour \texttt{/auth/disable\_2fa}.
  \item \textbf{Rotation de session à la connexion} et vidage explicite de session pour réduire les risques de fixation.
  \item \textbf{Réinitialisation de mot de passe} avec jetons signés mono-usage à durée courte (non implémenté, \S~\ref{chap:auth}).
  \item \textbf{Détection d'anomalies basée sur les journaux d'audit} (IP, ASN, pays), alertes et tableaux de bord; politiques de rétention/archivage.
  \item \textbf{Facteur résistant au phishing} (\emph{WebAuthn / Passkeys}) en option, en complément du TOTP.
  \item \textbf{Durcissement du pipeline email} : DKIM/DMARC stricts, liens à usage unique invalidés à la première visite.
\end{itemize}

\vspace{0.5em}
\noindent \textbf{Priorité court terme}. HTTPS forcé, CSP durcie, réauthentification pour désactiver la 2FA, chiffrement du secret TOTP, messages d'inscription non \og révélants \fg{}.

\section{Conclusion}
\noindent L'architecture d'authentification posée au \textbf{Chap.~\ref{chap:auth}} est saine: hachage robuste, 2FA réelle, captcha, limites et verrouillage. Ce chapitre a montré \emph{comment} ces mécanismes résistent aux attaques usuelles et a mis en lumière des durcissements concrets pour atteindre un niveau \og production \fg{}.

