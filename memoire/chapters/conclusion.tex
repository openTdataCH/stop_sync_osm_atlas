% !TeX spellcheck = fr_FR
\chapter*{Conclusion}
\addcontentsline{toc}{chapter}{Conclusion} % Adding toc entry


\section*{Synthèse du travail accompli}

La digitalisation croissante des services de mobilité a rendu la qualité et la cohérence des données de transport public plus cruciales que jamais. Ce projet est né du constat d'un \textbf{désalignement} persistant entre la base de données officielle des arrêts en Suisse, ATLAS, et la référence cartographique collaborative mondiale, OpenStreetMap (OSM). Ces divergences, qu'elles soient de nature géographique, nominative ou structurelle, engendrent des incohérences qui dégradent l'expérience des usagers et complexifient les opérations pour les planificateurs.

Pour répondre à cette problématique, nous avons conçu et mis en œuvre une solution complète en trois phases. La première phase a consisté à développer un \textbf{pipeline de traitement automatisé} en Python, appliquant une cascade de méthodes d'appariement par ordre de fiabilité décroissante : correspondance exacte par identifiants, par nom, par proximité géographique, et enfin par analyse des lignes de transport (GTFS et HRDF). Cette approche a permis de traiter l'intégralité des \textbf{55\,571} arrêts ATLAS.

Les résultats quantitatifs sont significatifs : nous avons réussi à établir \textbf{48\,419} correspondances, ce qui représente une couverture de \textbf{84,3\%} des arrêts ATLAS distincts. Ce processus a également permis de cataloguer et de prioriser des milliers d'anomalies, incluant plus de 10\,000 problèmes de distance et près de 15\,000 conflits d'attributs. Les 8\,416 arrêts ATLAS et 21\,980 nœuds OSM restés non appariés ont été isolés pour une analyse ciblée.

La deuxième phase a porté sur le développement d'une \textbf{application web full-stack} (Flask, SQLAlchemy, JavaScript) pour la validation humaine. Cette plateforme offre une interface cartographique interactive pour visualiser les données, un outil dédié pour trier et résoudre les problèmes détectés, et un mécanisme de \textbf{persistance des solutions}, assurant que les corrections manuelles sont conservées et réappliquées lors des futurs imports de données.

Enfin, la troisième phase a consisté à \textbf{sécuriser et conteneuriser} l'application. Nous avons implémenté un système d'authentification robuste (mots de passe hachés avec Argon2, authentification à deux facteurs, protection CSRF, limitation de débit) et déployé l'ensemble de la pile logicielle (backend, base de données MySQL, frontend) avec Docker, garantissant un déploiement reproductible et simplifié.

\section*{Bilan personnel et réflexif}

Au-delà de ses objectifs techniques, ce projet de fin d'études a constitué une expérience d'apprentissage exceptionnellement riche et transversale. Sur le plan technique, il m'a permis d'acquérir et de consolider des compétences dans des domaines variés : du traitement de données géospatiales complexes (ATLAS, GTFS, HRDF, OSM) à la conception d'algorithmes d'appariement, en passant par le développement d'une application web complète, la sécurisation d'API et le déploiement via la conteneurisation. Le défi consistant à rendre des données brutes et hétérogènes non seulement intelligibles mais aussi actionnables via une interface utilisateur intuitive a été particulièrement stimulant.

Sur le plan de l'organisation et de la gestion de projet, ce travail m'a confronté à la nécessité d'une démarche structurée, depuis l'analyse initiale du besoin jusqu'à la livraison d'un produit fonctionnel. La gestion des différentes facettes du projet – analyse de données, développement logiciel, rédaction technique – a exigé une planification rigoureuse et une capacité d'adaptation face aux difficultés rencontrées, que je considère comme autant d'opportunités d'apprentissage inhérentes au métier d'ingénieur.

Ce qui m'a le plus passionné fut sans doute la dimension tangible de ce travail. Explorer les jeux de données, visualiser les réseaux de transport sur une carte et identifier des anomalies concrètes m'a offert une nouvelle perspective sur la géographie et l'infrastructure de la Suisse. La visite des bureaux de CFF à Berne fut également une expérience très enrichissante, qui a permis de contextualiser l'importance de ce projet et de valider sa pertinence auprès d'experts du domaine.

\section*{Perspectives et améliorations futures}

Ce travail jette les bases d'un outil puissant, mais il ouvre également la voie à de nombreuses améliorations et extensions futures. Nous pouvons les regrouper en trois axes principaux.

Le premier axe concerne la \textbf{valorisation et l'impact à court terme}. La prochaine étape logique est d'établir un canal de communication formel avec la communauté OpenStreetMap suisse afin de valider et d'intégrer les corrections proposées. Parallèlement, un processus de rétroaction pourrait être mis en place pour signaler les incohérences avérées à l'équipe ATLAS. Pour pérenniser le projet, il serait également pertinent de le structurer pour une gouvernance open-source, en définissant des lignes directrices pour les contributions externes.

Le deuxième axe porte sur les \textbf{améliorations techniques}. Les algorithmes d'appariement pourraient être affinés, notamment en remplaçant l'approche séquentielle "hit-first" par un modèle de scoring (éventuellement basé sur l'apprentissage automatique) qui attribuerait un score de confiance à chaque correspondance potentielle. L'application web bénéficierait de fonctionnalités avancées, telles que la résolution de problèmes en masse ou des tableaux de bord analytiques pour suivre l'évolution de la qualité des données par région ou par opérateur.

Enfin, le troisième axe esquisse une \textbf{vision à moyen terme}. La méthodologie et l'outillage développés ici pourraient être étendus pour synchroniser d'autres types de données géographiques, en Suisse ou à l'étranger. L'application pourrait évoluer d'un outil de correction ponctuelle vers une plateforme de \textbf{surveillance continue de la qualité des données}, détectant automatiquement les nouvelles divergences au fil des mises à jour des sources. Un tel système pourrait devenir un pilier pour la gestion collaborative et durable de la qualité des données de transport public en Suisse.

