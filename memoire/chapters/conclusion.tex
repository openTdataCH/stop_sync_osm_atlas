% !TeX spellcheck = fr_FR
\chapter*{Conclusion}
\addcontentsline{toc}{chapter}{Conclusion} % Adding toc entry


Ce projet a cherché à approcher le problème de synchronisation des arrêts de transport public entre OSM et le système ATLAS en Suisse afin d'améliorer la précision des données de transport. Pour ce faire, nous avons mis en œuvre diverses méthodes de correspondance, telles que la correspondance exacte, par nom et par distance, tout en explorant la correspondance par itinéraire comme approche expérimentale. À ce jour, nous avons réussi à apparier 33 747 des 56 128 arrêts ATLAS. Par ailleurs, parmi les nœuds OSM sans correspondance actuelle, 27 759 sont associés à au moins un itinéraire, ce qui laisse entrevoir un potentiel significatif pour améliorer notre taux de correspondance grâce à l'intégration des données d'itinéraires.

Au-delà de ses aspects techniques, ce projet a constitué une expérience d'apprentissage particulièrement enrichissante. J'ai acquis des compétences en cartographie, dans les systèmes de routage des transports publics comme GTFS, ainsi qu'en développement web. Le défi de concevoir des algorithmes de correspondance et de rendre des données complexes accessibles m'a profondément stimulé, tout en mettant en lumière l'importance d'une communication claire dans les projets techniques. J’ai également trouvé fascinant d’explorer les cartes et de les examiner de près, car cela permet de mieux comprendre et apprécier le territoire suisse. Ce qui m'a le plus marqué, c'est le plaisir de découvrir ces domaines et de relever des défis qui, bien que complexes, se sont révélés passionnants.

En regardant vers l'avenir, plusieurs pistes d'amélioration se dessinent. On peut utiliser davantage d’informations disponibles, comme l’opérateur, parmi d’autres. Une analyse approfondie des balises et une importation prudente des nœuds OSM manquants s’imposent pour garantir l’intégrité des données. Un défi clé reste de déterminer, pour chaque correspondance, quel arrêt – OSM ou ATLAS – est le plus correct. Une solution envisagée serait d’améliorer notre application web et de l’ouvrir au public pour une vérification participative par des usagers familiers des zones concernées, ce qui nécessiterait d’importants efforts pour optimiser son ergonomie et ses fonctionnalités. Par ailleurs, la mise en place d’une structure de données robuste sera cruciale pour gérer et appliquer efficacement les corrections aux deux ensembles de données. Bien que le chemin à parcourir soit encore long, nos avancées actuelles renforcent notre détermination.
