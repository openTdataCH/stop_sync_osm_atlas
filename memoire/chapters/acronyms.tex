% !TeX spellcheck = fr_FR
% Acronym definitions
\chapter*{Acronyms}
\begin{description}
    \item[2FA] Two-Factor Authentication — authentification à deux facteurs.
    \item[API] Application Programming Interface — interface de programmation applicative.
    \item[ATLAS] Référentiel officiel suisse des arrêts (plateforme Open Transport Data Switzerland).
    \item[AWS] Amazon Web Services — plateforme de services cloud.
    \item[BBox] Bounding Box — rectangle englobant géographique.
    \item[CAPTCHA] Completely Automated Public Turing test to tell Computers and Humans Apart — test anti-robot.
    \item[CDN] Content Delivery Network — réseau de diffusion de contenu.
    \item[CFF] Chemins de fer fédéraux suisses (voir aussi SBB).
    \item[CSP] Content Security Policy — politique de sécurité de contenu.
    \item[CSRF] Cross-Site Request Forgery — falsification de requête inter-sites.
    \item[CSV] Comma-Separated Values — format de fichier tabulaire.
    \item[DOM] Document Object Model — modèle objet du document (navigateur).
    \item[DoS] Denial of Service — déni de service.
    \item[GTFS] General Transit Feed Specification — format d'échange pour horaires et arrêts.
    \item[HRDF] HAFAS Raw Data Format — format brut d'horaires (transport ferroviaire).
    \item[HTTP] HyperText Transfer Protocol — protocole de transfert hypertexte.
    \item[HTTPS] HyperText Transfer Protocol Secure — HTTP chiffré via TLS.
    \item[IP] Internet Protocol — protocole Internet (adresse IP).
    \item[JSON] JavaScript Object Notation — format d'échange de données.
    \item[KD-tree] k-dimensional tree — structure de données pour recherches de voisinage.
    \item[LRU] Least Recently Used — politique de cache « moins récemment utilisé ».
    \item[LV95] Système de coordonnées suisse CH1903+ / LV95.
    \item[ORM] Object–Relational Mapping — mappage objet-relationnel.
    \item[OSM] OpenStreetMap — projet cartographique collaboratif.
    \item[PDF] Portable Document Format — format de document portable.
    \item[PTv1] Public Transport version 1 — schéma OSM historique pour le transport public.
    \item[PTv2] Public Transport version 2 — schéma OSM moderne pour le transport public.
    \item[QR] Quick Response — code à réponse rapide (QR code).
    \item[RCE] Remote Code Execution — exécution de code à distance.
    \item[SARGable] Search ARGument able — requêtes exploitant directement des index SQL.
    \item[SBB] Schweizerische Bundesbahnen — Chemins de fer fédéraux suisses (voir aussi CFF).
    \item[SDK] Software Development Kit — kit de développement logiciel.
    \item[SES] (Amazon) Simple Email Service — service d'envoi d'emails.
    \item[SLOID] Identifiant d'arrêt ATLAS (identifiant interne des plateformes/quais).
    \item[SQL] Structured Query Language — langage de requêtes relationnelles.
    \item[SVG] Scalable Vector Graphics — format vectoriel pour le web.
    \item[TOTP] Time-based One-Time Password — mot de passe à usage unique basé sur le temps.
    \item[UIC] Union Internationale des Chemins de Fer — organisation internationale des chemins de fer (codes pays/gares).
    \item[UID/GID] User ID / Group ID — identifiants utilisateur et groupe sous Unix.
    \item[UI] User Interface — interface utilisateur.
    \item[URL] Uniform Resource Locator — localisateur uniforme de ressource.
    \item[WGS84] World Geodetic System 1984 — système géodésique mondial (coordonnées GPS).
    \item[XML] eXtensible Markup Language — langage de balisage extensible.
\end{description}